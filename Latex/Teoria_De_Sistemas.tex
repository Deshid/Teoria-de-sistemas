\documentclass[a4paper,12pt,numbers=noenddot]{scrreprt}
\setlength{\headheight}{61.24997pt} %Definition der KOMA-Textklasse
\usepackage{amsmath}
\usepackage{tikz} % Add this line to include the tikz package
\usepackage{fancyhdr}
\usepackage{graphicx}
\usepackage{ragged2e} % Add this line to include the ragged2e package
%\justifying % Add this command to justify the text
\usepackage{amsfonts}

%%%%%%%%%%%%%%%%%%%%%%
% Set up fancy header/footer
\pagestyle{fancy}
\fancyhead[LO,L]{Paula Labra}
\fancyhead[CO,C]{620438-2 Apunte}
\fancyhead[RO,R]{\today}
\fancyfoot[LO,L]{}
\fancyfoot[CO,C]{\thepage}
\fancyfoot[RO,R]{}
\renewcommand{\headrulewidth}{0.4pt}
\renewcommand{\footrulewidth}{0.4pt}
%%%%%%%%%%%%%%%%%%%%%%


\begin{document}

% Add the logo in the header
\lhead{\includegraphics[width=2cm]{LogoCat.png}} % Replace "logo.png" with the filename of your logo image

\title{Teoría de Sistemas}

% Contenido del documento
\section*{Concepto de señal}
    \begin{itemize}
    \item \textbf{Señal:} Es una función de una o más variables 
    independientes que contiene información acerca de la naturaleza de un fenómeno [Haykin].
    \begin{center}
    \begin{tikzpicture}
        % Define the x and y coordinates
        \coordinate (A) at (0,0);
        \coordinate (X) at (0.25,0.5);
        \coordinate (Y) at (0.5,-1);
        \coordinate (Z) at (1,0.5);
        \coordinate (B) at (0.5,1);
        \coordinate (C) at (1,-1);
        \coordinate (D) at (1.5,0);
        \coordinate (E) at (2,1);
        \coordinate (F) at (2.5,-1);
        \coordinate (G) at (3,0);
        
        % Draw the axes
        \draw[->] (-0.5,0) -- (3.5,0) node[right] {$t$ (tiempo)};
        \draw[->] (0,-1.5) -- (0,1.5) node[above] {$x(t)$};
        
        % Plot the functions
        \draw[blue, thick] (A) sin (B) cos (C) sin (D) cos (E) sin (F) cos (G);
        \draw[red, thick] (A) sin (B) cos (C) sin (D) cos (E) sin (F) cos (G);
        \draw[green, thick] (A) sin (B) cos (C) sin (D) cos (E) sin (F) cos (G);

    \end{tikzpicture}
    \end{center}
    \end{itemize}

\section*{Notas extras}
\begin{itemize}
    \item[] \textbf{Nota:} Cualquier cosa que varía en el tiempo es una señal.
\end{itemize}
    
\section*{Concepto de sistema}
    \begin{itemize}
    \item \textbf{Sistema: [Haykin]} Conjunto de elementos que interactúan 
    entre sí para lograr un objetivo.
    \item[]\textbf{Otra definición:} Entidad que manipula una o más señales para llevar a 
    cabo una función, produciendo de ese modo nuestras señales.
    \item \textbf{Sistema: [puente]} Conjunto de elementos físicos o abstractos
    relacionados entre si de forma que modificaciones o alteraciones en determinadas
    magnitudes (variables, señales) de uno de ellos puedan inferir o ser influidos
    las de los demás. 
\end{itemize}

\section*{Representaciones interna y externa}
\begin{itemize}
    \item \textbf{Representación externa:} Análisis a partir de las manifestaciones
    externas del sistema. Filosofía de \textacutedbl la caja negra\textacutedbl. Relación
    entrada/salida. (Función de transferencia - misma entrada, distinta salida)
\end{itemize}

\begin{equation*}
    f(u, y, \dot{u}, \dot{y}, \ddot{u}, \ddot{y}, ...) = 0
\end{equation*} \\
\textbf{Donde:} $u$ es la entrada, $y$ es la salida, $\dot{u}$ es la primera 
derivada de la entrada, $\dot{y}$ es la primera derivada de la salida, $\ddot{u}$ 
es la segunda derivada de la entrada, $\ddot{y}$ es la segunda derivada de la salida, etc. \\
\\
Si nos fijamos, la ecuación anterior es una ecuación diferencial. Lo que significa
que la primera derivada representa la velocidad, y la segunda derivada representa 
la aceleración.

\begin{itemize}
\item \textbf{Representación interna:} descripción del sistema a través de
variables internas denominadas \textbf{variables de estado}: conjunto de variables
tales que siendo conocidas para $t=t_0$, y conocida la entrada $t \geq t_0$, 
permite obtener la salida para todo 
$t \geq t_0$. \\
\end{itemize}

\begin{equation*}
    \dot{X}_i = f_i(x_1, x_2, ... , x_n, u)
\end{equation*} 
\begin{equation*}
    y = g(x_1, x_2, ... , x_n, u)
\end{equation*} \\

\begin{center}
\begin{tikzpicture}

    % Draw the rectangle
    \draw (0,0) rectangle (4,2);
    
    % Add the text "Sistema" in the middle
    \node at (2,1) {Sistema};

    \node at (-3.5,1) {Entradas};
    \node at (-2,1.5) {$u_1$};
    \node at (-2,1.25) {$u_2$};
    \node at (-2,1) {$u_3$};
    \node at (-2,0.77) {$...$};
    \node at (-2,0.5) {$u_p$};

    \node at (7.5,1) {Salidas};
    \node at (6,1.5) {$y_1$};
    \node at (6,1.25) {$y_2$};
    \node at (6,1) {$y_3$};
    \node at (6,0.77) {$...$};
    \node at (6,0.5) {$y_q$};

    \node at (3,4.5) {Perturbaciones};
    \node at (2,4) {$z_1$};
    \node at (2.4,4) {$z_2$};
    \node at (2.8,4) {$z_3$};
    \node at (3.2,4) {$...$};
    \node at (3.6,4) {$z_r$};
    \node[font=\tiny, text width=2.5cm] at (2,0.5) {Variables de estado: $x_1, x_2, ..., x_n$};

    % Add arrows on the left side
    \draw[->] (-1.5,1.5) -- (0,1.5);
    \draw[->] (-1.5,1.25) -- (0,1.25); 
    \draw[->] (-1.5,1) -- (0,1);
    \draw[->] (-1.5,0.77) -- (0,0.77);
    \draw[->] (-1.5,0.5) -- (0,0.5);
    
    % Add arrows on the right side
    \draw[->] (4,1) -- (5.5,1);
    \draw[->] (4,0.5) -- (5.5,0.5);
    \draw[->] (4,0.77) -- (5.5,0.77);
    \draw[->] (4,1.5) -- (5.5,1.5);
    \draw[->] (4,1.25) -- (5.5,1.25);
    
    % Add arrows on the top
    \draw[->] (2,3.5) -- (2,2);
    \draw[->] (2.4,3.5) -- (2.4,2);
    \draw[->] (2.8,3.5) -- (2.8,2); 
    \draw[->] (3.2,3.5) -- (3.2,2);
    \draw[->] (3.6,3.5) -- (3.6,2);
    
\end{tikzpicture}
\end{center}

OBS: Las Perturbaciones son señales o "cosas" que afectan al sistema, 
pero que no son controlables.

\section*{Tipos de sistemas}
\begin{itemize}
    \item {En bucle abierto / Realimentados}
    \item {Lineales / No lineales}
    \item {De parámetros concentrados / Distribuidos}
    \item {Estacionarios / Variantes}
    \item {Deterministas / Estocásticos}
    \item {Monovariables / Multivariables}
    \item {De tiempo continuo / Discreto}
\end{itemize}

\section*{Sistemas en bucle abierto}
\begin{itemize}
    \item \textbf{Bucle abierto:} La señal de entrada actúa directamente
    sobre el controlador del sistema.
    \item[] \textbf{Ejemplo de elementos de control:} cargador de celular,
    zona de nuestro cerebro.
\end{itemize}


\begin{center}
\begin{tikzpicture}

% Draw the rectangle
\draw (0,0) rectangle (3,2);
\draw (0,0) rectangle (3,2);
% Add the text in the middle
\node[text width=2.5cm, align=center] at (1.5,1) {Elementos de control};
\draw (7.5,0) rectangle (4.5,2);
\draw (7.5,0) rectangle (4.5,2);
\node[text width=2.5cm, align=center] at (6,1) {Planta o proceso};


\node at (-2.3,1.5) {Entrada};
\draw[->] (-1.5,1) -- (0,1);
\draw[->] (3,1) -- (4.5,1);
\node at (9.5,1.5) {Salida};
\draw[->] (7.5,1) -- (9,1);

\end{tikzpicture}
\end{center}

\section*{Sistemas realimentados}
\begin{itemize}
    \item \textbf{Bucle cerrado (realimentados):} La señal de entrada, antes de ser
    introducida en el controlador, es modificada en función de la salida.
\end{itemize}


\begin{center}
    \begin{tikzpicture}
    
    % Draw the rectangle
    \draw (0,0) rectangle (3,2);
    \draw (0,0) rectangle (3,2);
    % Add the text in the middle
    \node[text width=2.5cm, align=center] at (1.5,1) {Elementos de control};
    \draw (7.5,0) rectangle (4.5,2);
    \draw (7.5,0) rectangle (4.5,2);
    \node[text width=2.5cm, align=center] at (6,1) {Planta o proceso};
    \draw (5,-3) rectangle (2,-1);
    \draw (5,-3) rectangle (2,-1);
    \node[text width=4cm, align=center] at (3.5,-2) {Elementos de realimentación};
    
    \draw (-1.79,1) circle (0.3);
    \node at (-3,1.5) {Entrada +};
    \node at (-2.2,0.5) {-};
    \draw[->] (-4,1) -- (-2.1,1);
    \draw[->] (-1.5,1) -- (0,1);
    \draw[->] (3,1) -- (4.5,1);
    \node at (9.5,1.5) {Salida};
    \draw[->] (7.5,1) -- (9,1);
    \draw[->] (8.5,1) -- (8.5,-2) -- (5,-2) -- (5,-2);
    \draw[->] (2,-2) -- (2,-2) -- (-1.8,-2) -- (-1.8,0.7);
    
    \end{tikzpicture}
    \end{center}

    \newpage
    \textbf{Esquema típico de control. Ejemplo de la ducha:} el grifo del
    agua caliente está abierto al máximo. Ajustamos la temperatira del agua con el grifo
    de agua fría.

    \begin{center}
    \includegraphics[width=17cm]{Diagrama sin título-Página-2.png}
    \end{center}
    \textbf{Ejemplo de perturbación:} alguien abre el grifo de agua caliente en otra parte 
    de la casa, llega menos agua caliente a la ducha y la mezcla se enfría. Gracias a la realimentación,
    el cerebro detecta la nueva situación y da la orden de cerrar un poco el grifo de agua fría.
    
    \section*{Sistemas lineales / no lineales}
    Los sistemas lineales son aquellos que cumplen con el \textbf{principio de superposición}.

    Si
    \begin{equation*}
        u_1(t) \rightarrow y_1(t)
    \end{equation*}

    \begin{equation*}
        u_2(t) \rightarrow y_2(t)
    \end{equation*}

    Entonces
    \begin{equation*}
        \alpha u_1(t) + \beta u_2(t) \rightarrow \alpha y_1(t) + \beta y_2(t)
    \end{equation*}
    \begin{equation*}
        \alpha, \beta \in \mathbb{R}
    \end{equation*}

    \section*{Sistemas de parámetros concentrados / distribuidos}
    \begin{itemize}
        \item \textbf{Parámetros concentrados:} Los parámetros del sistema son constantes
        en el tiempo y en el espacio.
        \item \textbf{Parámetros distribuidos:} Los parámetros del sistema son variables
        en el tiempo y en el espacio.
    \end{itemize}

    \section*{Sistemas estacionarios / variantes}
    \begin{itemize}
        \item \textbf{Estacionarios:} Los parámetros del sistema son constantes en el tiempo.
        A la misma entrada en distintos instantes, responden igual.
        \item \textbf{Variantes:} Los parámetros o los comportamientos del sistema 
        son variables en el tiempo.
    \end{itemize}

    \section*{Sistemas deterministas / estocásticos}
    \begin{itemize}
        \item \textbf{Deterministas:} Si conocemos la entrada, conocemos la salida.
        \item \textbf{Estocásticos:} Si conocemos la entrada, no podemos predecir la salida.
    \end{itemize}

    \section*{Sistemas monovariables / multivariables}
    \begin{itemize}
        \item \textbf{Monovariables:} Un solo tipo de entrada y un solo tipo de salida.
        \textbf{SISO}(Single Input Single Output).
        \item \textbf{Multivariables:} Más de un tipo de entrada \textbf{MISO}(Multiple Input 
        Single Output) y más de un tipo de salida \textbf{SIMO}(Single Input Multiple Output) o
        ambas \textbf{MIMO}(Multiple Input Multiple Output).
    \end{itemize}

    \section*{Sistemas continuos / discretos}
    \begin{itemize}
        \item \textbf{Continuos:} Las señales de entrada y salida son variables continuas en el tiempo.
        \item \textbf{Discretos:} Las señales de entrada y salida son discretas en el tiempo.
        Suelen ser resultado de un \textbf{muestreo} de señales continuas.
    \end{itemize}


    \begin{minipage}[t]{0.45\textwidth}
        \begin{tikzpicture}[scale=0.5]
            % Define the x and y coordinates
            \coordinate (A) at (0,0);
            \coordinate (B) at (1,0);
            \coordinate (C) at (2,0);
            \coordinate (D) at (3,0);
            \coordinate (E) at (4,0);
            \coordinate (F) at (5,0);
            \coordinate (G) at (6,0);
            
            % Draw the axes
            \draw[->] (-0.5,0) -- (6.5,0) node[right] {$t$ (continuo)};
            \draw[->] (0,-1.5) -- (0,1.5) node[above] {$x[t]$};
            
            % Plot the function
            \draw[blue, thick, samples=100, domain=0:6, smooth] plot (\x, {sin(\x r)});
        \end{tikzpicture}
    \end{minipage}
    \begin{minipage}[t]{0.45\textwidth}
        \begin{tikzpicture}[scale=0.5]
            % Define the x and y coordinates
            \coordinate (A) at (0,0);
            \coordinate (B) at (1,0);
            \coordinate (C) at (2,0);
            \coordinate (D) at (3,0);
            \coordinate (E) at (4,0);
            \coordinate (F) at (5,0);
            \coordinate (G) at (6,0);
            
            % Draw the axes
            \draw[->] (-0.5,0) -- (6.5,0) node[right] {$k$ (discreto)};
            \draw[->] (0,-1.5) -- (0,1.5) node[above] {$x[k]$};
            
            % Plot the points
            \foreach \x in {0,1,2,3,4,5,6}
            \filldraw[blue] (\x, {sin(\x r)}) circle (2pt);
        \end{tikzpicture}
    \end{minipage}
    
\newpage
    \section*{Ecuaciones diferenciales y dinámica}


\end{document}
